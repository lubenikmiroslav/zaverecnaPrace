\documentclass[12pt, a4paper,
twoside, openright
]{report}

\usepackage[utf8]{inputenc}
\usepackage[czech]{babel}
\usepackage[T1]{fontenc}
\usepackage{cmap}
\usepackage{graphicx}
\usepackage{subcaption}
\usepackage{hyperref}
\usepackage{listings}
\usepackage{xcolor}
\usepackage{float}
\usepackage{amsmath}
\usepackage[top=2.5cm, bottom=2.5cm, inner=3.5cm, outer=1.5cm]{geometry}
\usepackage{array}
\usepackage{longtable}
\usepackage{caption}
\usepackage{multicol}
\usepackage{fancyhdr}
\usepackage{needspace}
\usepackage{placeins}
\usepackage{tocloft}
\usepackage[pagestyles]{titlesec}
\setlength{\headheight}{15pt}

% Profesionální typografie
\linespread{1.25}
\setlength{\parskip}{0.5em}
\setlength{\parindent}{1.5cm}

% Formátování kapitol podle šablony
\titleformat{\chapter}[block]{\scshape\bfseries\LARGE}{\thechapter}{10pt}{\vspace{0pt}}[\vspace{-22pt}]
\titleformat{\section}[block]{\scshape\bfseries\Large}{\thesection}{10pt}{\vspace{0pt}}
\titleformat{\subsection}[block]{\bfseries\large}{\thesubsection}{10pt}{\vspace{0pt}}

% Nastavení obsahu
\setlength{\cftbeforechapskip}{0pt}
\setlength{\cftbeforesecskip}{0pt}
\setcounter{secnumdepth}{2}
\setcounter{tocdepth}{1}

% Nastavení cesty k obrázkům
\graphicspath{{./}{prilohy/screenshots/}{../assets/}}

% Nastavení pro kód
\lstset{
    language=Dart,
    basicstyle=\ttfamily\small,
    keywordstyle=\color{blue}\bfseries,
    commentstyle=\color{green!60!black},
    stringstyle=\color{red},
    numbers=left,
    numberstyle=\tiny\color{gray},
    stepnumber=1,
    numbersep=5pt,
    backgroundcolor=\color{gray!10},
    frame=single,
    breaklines=true,
    breakatwhitespace=true,
    tabsize=2,
    showspaces=false,
    showstringspaces=false,
    showtabs=false,
    captionpos=b
}

% Nastavení záhlaví a zápatí
\pagestyle{fancy}
\fancyhf{}
\fancyhead[C]{\small Závěrečná studijní práce -- Miroslav Lubeník -- IT4 -- 2025/2026}
\fancyfoot[C]{\thepage}
\renewcommand{\headrulewidth}{0.025pt}
\renewcommand{\footrulewidth}{0pt}

\begin{document}

\pagestyle{empty}
\pagenumbering{arabic}
\setcounter{page}{1}

% Titulní stránka s informacemi (podle šablony)
{\fontfamily{phv}\selectfont
    %% Logo školy
    \begin{figure}[h]
        \centering
        \includegraphics[width=0.6\linewidth]{latex/image/logo-skoly.png} 
    \end{figure}
    
    
    %% Hlavička práce a její název
    {\bfseries
        \begin{center}
            \vspace{0.025 \textheight}
            \LARGE{ZÁVĚREČNÁ STUDIJNÍ PRÁCE}\\
            \large{dokumentace}\\
            \vspace{0.05 \textheight}
            \begin{figure}[h]
                \centering
                \includegraphics[width=0.3\linewidth]{logo_aplikace.png} 
            \end{figure}
            \vspace{0.025 \textheight}
            \LARGE{HabitTrack -- Aplikace pro sledování návyků a zdraví}\\
        \end{center}  
    }
    
    \vspace{0.02 \textheight}
    \begin{table}[h!]
        \begin{tabular}{ll}
            \textbf{Autor:} & Miroslav Lubeník\\ 
            \textbf{Obor:} & 18-20-M/01 INFORMAČNÍ TECHNOLOGIE\\
            \textbf{} & se zaměřením na počítačové sítě a programování\\
            \textbf{Třída:} & IT4\\
            \textbf{Školní rok:} & 2025/2026\\
        \end{tabular}
        
    \end{table}
}
\cleardoublepage

% Poděkování
\newpage
\thispagestyle{empty}
\section*{Poděkování}
\textit{poděkování (například vedoucímu práce).}

\vspace{2cm}
\dotfill

% Prohlášení
\newpage
\thispagestyle{empty}
\section*{Prohlášení}
Prohlašuji, že jsem závěrečnou práci vypracoval samostatně a uvedl veškeré použité informační zdroje.

Souhlasím, aby tato studijní práce byla použita k~výukovým účelům na~Střední průmyslové a~umělecké škole v~Opavě, Praskova 399/8.

\vspace{2cm}
V~Opavě \quad \today

\vspace{1cm}
\dotfill

\textit{podpis autora práce}

% Abstrakt (česky i anglicky pod sebe na jedné stránce)
\newpage
\noindent{\Large{\bfseries{Abstrakt}\\}}
\noindent Tato závěrečná práce se zabývá vytvořením mobilní aplikace HabitTrack pro~sledování návyků a~zdraví pomocí Flutter frameworku. Aplikace umožňuje uživatelům vytvářet a~spravovat vlastní návyky, sledovat jejich denní plnění, zobrazovat statistiky pokroku a~personalizovat prostředí aplikace. Aplikace je navržena s~důrazem na~moderní uživatelské rozhraní, offline fungování a~intuitivní ovládání. Práce obsahuje popis teoretických základů, použité technologie, způsoby řešení a~výsledky implementace.

\vspace{12pt}

\noindent{\large{\bfseries{Klíčová slova}}}
\noindent mobilní aplikace, sledování návyků, HabitTrack, personalizace, statistiky

\vspace{18pt}

\noindent{\Large{\bfseries{Abstract}}}
\noindent This final thesis deals with the creation of a~mobile application HabitTrack for~tracking habits and health using the Flutter framework. The application allows users to~create and manage their own habits, track their daily completion, display progress statistics, and personalize the application environment. The application is designed with emphasis on~modern user interface, offline functionality, and intuitive operation. The thesis contains description of~theoretical foundations, used technologies, solution approaches and implementation results.

\vspace{12pt}

\noindent{\large{\bfseries{Keywords}}}
\noindent mobile application, habit tracking, HabitTrack, personalization, statistics

% Obsah
\newpage
\tableofcontents

% Úvod (nečíslovaný)
\chapter*{Úvod}
\addcontentsline{toc}{chapter}{Úvod}

Tato závěrečná práce se zabývá vytvořením mobilní aplikace HabitTrack pro~sledování návyků a~zdraví. Aplikace poskytuje uživatelům moderní a~intuitivní nástroj pro~správu jejich denních návyků s~důrazem na~vizualizaci pokroku a~motivaci k~dlouhodobé pravidelnosti.

Sledování návyků je důležitou součástí osobního rozvoje a~zdravého životního stylu. Mobilní aplikace mohou uživatelům významně pomoci při budování a~udržování pozitivních návyků prostřednictvím vizualizace pokroku, připomenutí a~gamifikace. Aplikace HabitTrack umožňuje uživatelům vytvářet vlastní návyky, sledovat jejich denní plnění, zobrazovat statistiky pokroku a~personalizovat prostředí aplikace podle svých preferencí.

Volba této problematiky byla motivována potřebou praktické aplikace pro~osobní použití, zájmem o~vývoj mobilních aplikací s~komplexní funkcionalitou a~snahou vytvořit aplikaci s~moderním designem a~uživatelsky přívětivým rozhraním, která pomůže uživatelům udržet si pozitivní návyky a~sledovat jejich pokrok.

Hlavní cíle této závěrečné práce jsou: navrhnout a~implementovat mobilní aplikaci pro~sledování návyků, vytvořit intuitivní uživatelské rozhraní s~moderním designem, implementovat systém pro~ukládání dat s~offline podporou, zajistit personalizaci aplikace (dark mode, barvy, profil), vytvořit systém statistik a~vizualizace pokroku a~implementovat notifikace pro~připomenutí o~návycích.

Práce je rozdělena do~následujících kapitol. Kapitola 1 -- Teoretická a~metodická východiska popisuje problematiku sledování návyků a~metodiky pro~jejich efektivní sledování. Kapitola 2 -- Využité technologie představuje konkrétní technologie, nástroje a~knihovny použité při vývoji aplikace. Kapitola 3 -- Způsoby řešení a~použité postupy popisuje návrh architektury aplikace, databázového schématu a~implementační postupy. Kapitola 4 -- Výsledky řešení, výstupy, uživatelský manuál prezentuje finální aplikaci, její funkcionality a~uživatelský manuál. Závěr shrnuje výsledky práce a~navrhuje možná vylepšení.

% Kapitola 1: Teoretická a metodická východiska
\chapter{Teoretická a metodická východiska}

\section{Problematika sledování návyků}

Sledování návyků je důležitou součástí osobního rozvoje a~zdravého životního stylu. Návyky představují opakované chování, které se stává automatickým a~pomáhá jedincům dosahovat dlouhodobých cílů. Efektivní sledování návyků umožňuje uživatelům identifikovat vzorce chování, měřit pokrok a~udržovat motivaci k~pravidelnému plnění stanovených cílů.

Moderní mobilní aplikace mohou uživatelům významně pomoci při budování a~udržování pozitivních návyků prostřednictvím vizualizace pokroku, připomenutí a~gamifikace. Aplikace HabitTrack byla navržena s~důrazem na~tyto aspekty, aby poskytla uživatelům komplexní nástroj pro~správu jejich návyků.

\section{Metodika sledování návyků}

Pro~efektivní sledování návyků je důležité stanovit jasné cíle, pravidelně zaznamenávat pokrok a~vizualizovat výsledky. Aplikace HabitTrack implementuje několik klíčových principů:

\begin{itemize}
    \item \textbf{Každodenní sledování} -- uživatelé mohou označit splnění návyku pro~konkrétní den.
    \item \textbf{Vizualizace pokroku} -- kalendář s~historií plnění a~statistiky poskytují přehled o~pokroku.
    \item \textbf{Gamifikace} -- systém achievementů motivuje uživatele k~pravidelnému plnění návyků.
    \item \textbf{Připomenutí} -- notifikace pomáhají uživatelům nezapomenout na~plnění návyků.
    \item \textbf{Personalizace} -- možnost nastavení barev, dark mode a~profilové fotky zvyšuje uživatelskou zkušenost.
\end{itemize}

\section{Architektura mobilních aplikací}

Pro~vytvoření funkční mobilní aplikace je důležité navrhnout správnou architekturu, která zajistí oddělení logiky aplikace od~prezentační vrstvy, umožní snadnou údržbu a~rozšíření funkcionalit. Aplikace HabitTrack využívá čistou architekturu s~rozdělením na~vrstvy pro~data, logiku a~prezentaci.

\section{Offline fungování aplikace}

Důležitým aspektem mobilní aplikace pro~sledování návyků je možnost fungování bez~síťového připojení. Uživatelé potřebují mít přístup ke~svým datům kdykoli a~kdekoli, bez~ohledu na~dostupnost internetového připojení. Aplikace HabitTrack ukládá všechna data lokálně na~zařízení, což zajišťuje rychlý přístup a~neustálou dostupnost funkcionalit.

% Kapitola 2: Využité technologie
\chapter{Využité technologie}

\section{Flutter Framework}

Aplikace byla vyvinuta pomocí Flutter frameworku \cite{flutter}, který umožňuje vytvářet cross-platform mobilní aplikace z~jednoho zdrojového kódu. Flutter používá programovací jazyk Dart \cite{dart} a~kompiluje aplikace do~nativního kódu, což zajišťuje vysoký výkon.

\section{Dart programovací jazyk}

Dart je objektově orientovaný programovací jazyk \cite{dart}, který poskytuje statickou typovou kontrolu, asynchronní programování a~automatickou správu paměti.

\section{SQLite databáze}

Pro~lokální ukládání dat byla použita SQLite databáze přes balíček \texttt{sqflite} \cite{sqflite}, která umožňuje offline fungování aplikace a~rychlý přístup k~datům.

\section{Další použité knihovny}

Aplikace využívá další knihovny pro~konkrétní funkcionality: \texttt{shared\_preferences} \cite{shared_preferences} pro~ukládání nastavení, \texttt{flutter\_local\_notifications} \cite{flutter_local_notifications} pro~notifikace, \texttt{fl\_chart} \cite{fl_chart} pro~vytváření grafů a~\texttt{image\_picker} pro~výběr obrázků z~galerie.

\section{Path Provider}

Balíček \texttt{path\_provider} poskytuje přístup k~systémovým cestám pro~ukládání souborů. V~aplikaci je použit pro~ukládání profilových fotografií uživatelů na~správné místo v~systému souborů.

\section{Zdůvodnění výběru technologií}

Výběr technologií byl proveden s~ohledem na~požadavek na~cross-platform kompatibilitu, potřebu offline fungování aplikace, snahu o~moderní a~atraktivní uživatelské rozhraní, dostupnost a~kvalitu dokumentace a~aktivitu komunity a~podporu. Při výběru technologií byly využity zdroje jako oficiální dokumentace \cite{flutter_docs, dart_docs}, Flutter Cookbook \cite{flutter_cookbook}, komunita na~Stack Overflow \cite{stackoverflow} a~GitHub \cite{github} pro~inspiraci z~open source projektů. Všechny použité technologie jsou open-source a~bezplatné, což umožňuje volné použití a~distribuci aplikace.

% Kapitola 3: Způsoby řešení a použité postupy
\chapter{Způsoby řešení a použité postupy}

\section{Návrh architektury aplikace}

Aplikace je strukturována podle principů čisté architektury \cite{clean_architecture} s~oddělením vrstev. Struktura projektu je následující:

\subsection{Struktura projektu}

\begin{itemize}
    \item \texttt{lib/screens/} -- obrazovky aplikace (UI vrstva).
    \item \texttt{lib/services/} -- business logika a~služby (databáze, notifikace).
    \item \texttt{lib/widgets/} -- znovupoužitelné widgety.
    \item \texttt{lib/styles/} -- centralizované styly a~design systém.
\end{itemize}

\subsection{Design Patterns}

V~aplikaci jsou použity následující návrhové vzory \cite{design_patterns}:

\begin{itemize}
    \item \textbf{Singleton} -- pro~DatabaseHelper a~NotificationService (zajišťuje jedinou instanci).
    \item \textbf{State Management} -- pomocí StatefulWidget a~setState pro~lokální stav.
    \item \textbf{Repository Pattern} -- pro~práci s~databází (abstrakce datové vrstvy).
    \item \textbf{Factory Pattern} -- pro~vytváření widgetů a~stylů.
\end{itemize}

\section{Návrh databáze}

Databáze obsahuje následující tabulky:

\subsection{Schéma databáze}

\begin{itemize}
    \item \texttt{users} -- informace o~uživatelích (id, email, nickname, password\_hash, profile\_photo\_path)
    \item \texttt{habits} -- definice návyků (id, user\_id, name, description, color, icon, daily\_target, atd.)
    \item \texttt{habit\_logs} -- záznamy o~plnění návyků (id, habit\_id, date, completed)
    \item \texttt{calendar\_notes} -- poznámky k~datům (id, user\_id, date, note)
    \item \texttt{achievements} -- odemčené achievementy (id, user\_id, habit\_id, type, unlocked\_at)
\end{itemize}

\subsection{Migrace databáze}

Databáze podporuje migrace pro~plynulé aktualizace schématu. Při změně verze databáze se automaticky provedou potřebné změny (přidání sloupců, vytvoření nových tabulek).

\section{Návrh uživatelského rozhraní}

UI bylo navrženo s~důrazem na:

\begin{itemize}
    \item Minimalistický a~moderní design.
    \item Konzistentní barevné schéma (gradient Pink → Orange).
    \item Animace pro~lepší uživatelský zážitek.
    \item Dark mode podpora.
    \item Responzivní layout pro~různé velikosti obrazovek.
\end{itemize}

\section{Implementační postupy}

\subsection{Inicializace aplikace}

Aplikace začíná v~souboru \texttt{main.dart}, kde se inicializují klíčové služby (databáze, notifikace) a~načítají uživatelská nastavení.

\begin{lstlisting}[caption=Inicializace aplikace v~main.dart, label=lst:main]
void main() async {
  WidgetsFlutterBinding.ensureInitialized();
  await DatabaseHelper.instance.database;
  await NotificationService.instance.initialize();

  final prefs = await SharedPreferences.getInstance();
  final savedEmail = prefs.getString('user_email');
  final isDarkMode = prefs.getBool('dark_mode') ?? false;
  final themeColor = prefs.getString('theme_color') ?? '#009688';

  runApp(HabitTrackApp(
    isLoggedIn: savedEmail != null,
    isDarkMode: isDarkMode,
    themeColor: themeColor,
  ));
}
\end{lstlisting}

\subsection{Databázová vrstva}

Databázová vrstva je implementována pomocí Singleton patternu v~třídě \texttt{DatabaseHelper}. Všechny databázové operace jsou asynchronní a~používají SQL dotazy.

\begin{lstlisting}[caption=Implementace Singleton patternu pro~DatabaseHelper, label=lst:database]
class DatabaseHelper {
  static final DatabaseHelper instance = DatabaseHelper._init();
  static Database? _database;

  DatabaseHelper._init();

  Future<Database> get database async {
    if (_database != null) return _database!;
    _database = await _initDB('habittrack.db');
    return _database!;
  }

  Future<Database> _initDB(String filePath) async {
    final dbPath = await getDatabasesPath();
    final path = join(dbPath, filePath);
    return await openDatabase(path, version: 6, 
                              onCreate: _createDB, 
                              onUpgrade: _onUpgrade);
  }
}
\end{lstlisting}

\subsection{Správa návyků}

Hlavní obrazovka zobrazuje seznam návyků uživatele s~možností označit je jako splněné. Při dokončení návyku se zobrazí confetti efekt pro~vizuální zpětnou vazbu.

\begin{lstlisting}[caption=Načítání návyků a~jejich statistik, label=lst:habits]
Future<void> _loadHabits() async {
  final data = await DatabaseHelper.instance.getHabits(userId);
  final todayStr = DateTime.now().toIso8601String().split('T').first;
  
  Map<int, int> todayCount = {};
  Map<int, int> streaks = {};
  
  for (var habit in data) {
    final habitId = habit['id'] as int;
    final count = await DatabaseHelper.instance
        .getDailyCompletionCount(habitId, todayStr);
    todayCount[habitId] = count;
    
    final streak = await DatabaseHelper.instance
        .getHabitStreak(habitId);
    streaks[habitId] = streak;
  }
  
  setState(() {
    habits = data;
    completedTodayCount = todayCount;
    habitStreaks = streaks;
  });
}
\end{lstlisting}

\subsection{Stylování a~design systém}

Aplikace používá centralizovaný design systém s~oddělenými soubory pro~barvy, gradienty, textové styly a~dekorace. To umožňuje snadnou údržbu a~konzistenci vzhledu.

\begin{lstlisting}[caption=Definice barev v~design systému, label=lst:colors]
class AppColors {
  // Primární barvy
  static const Color primaryOrange = Color(0xFFFF9800);
  static const Color primaryPink = Color(0xFFE91E63);
  static const Color primaryPurple = Color(0xFF9C27B0);
  
  // Barvy pro~návyky
  static const Color habitPink = Color(0xFFE91E63);
  static const Color habitRed = Color(0xFFF44336);
  static const Color habitOrange = Color(0xFFFF9800);
  // ... další barvy
}
\end{lstlisting}

\subsection{Animace a~efekty}

Aplikace obsahuje různé animace pro~zlepšení uživatelského zážitku:
\begin{itemize}
    \item AnimatedGradientBackground -- animované gradientní pozadí.
    \item ConfettiEffect -- confetti při dokončení návyku.
    \item BounceAnimation -- bounce efekt pro~seznamy.
    \item ScaleOnTap -- scale efekt při kliknutí.
    \item SkeletonLoader -- skeleton loading screens.
\end{itemize}

\section{Způsoby testování}

Aplikace byla testována na~několika úrovních:

\subsection{Manuální testování}

Aplikace byla manuálně testována na:
\begin{itemize}
    \item Android emulátoru (různé verze Android).
    \item Fyzickém Android zařízení.
\end{itemize}

\subsection{Testované funkcionality}

\begin{itemize}
    \item Autentizace (registrace, přihlášení).
    \item Správa návyků (vytváření, úprava, mazání).
    \item Označení návyku jako splněného.
    \item Zobrazení statistik a~grafů.
    \item Notifikace a~připomenutí.
    \item Dark mode přepínání.
    \item Personalizace (barvy, profilová fotka).
\end{itemize}

\section{Řešení problémů}

Během vývoje byly řešeny následující problémy:

\subsection{Přetečení layoutu na~kartách návyků}

\textbf{Problém:} Ikonky a~text se překrývaly na~kartách návyků při menších obrazovkách.

\textbf{Řešení:} Zvýšena výška karet a~upraveno rozložení prvků s~lepším paddingem.

\subsection{Bílá obrazovka před splash screenem}

\textbf{Problém:} Na~Android zařízeních se zobrazovala bílá obrazovka před načtením Flutter splash screenu.

\textbf{Řešení:} Upraveny native Android launch screen soubory (\texttt{launch\_background.xml}) pro~zobrazení gradientního pozadí.

\subsection{Type mismatch při dark mode}

\textbf{Problém:} Některé barvy byly definovány jako funkce vyžadující BuildContext, ale používaly se jako statické hodnoty.

\textbf{Řešení:} Rozděleny barvy na~statické (pro~gradient pozadí) a~context-aware (pro~karty a~text).

% Kapitola 4: Výsledky řešení, výstupy, uživatelský manuál
\chapter{Výsledky řešení, výstupy, uživatelský manuál}

\section{Splněné cíle práce}

Všechny stanovené cíle práce byly úspěšně splněny:

\begin{itemize}
    \item ✓ Navržena a~implementována mobilní aplikace pro~sledování návyků.
    \item ✓ Vytvořeno intuitivní uživatelské rozhraní s~moderním designem.
    \item ✓ Implementován systém pro~ukládání dat s~offline podporou.
    \item ✓ Zajištěna personalizace aplikace (dark mode, barvy, profil).
    \item ✓ Vytvořen systém statistik a~vizualizace pokroku.
    \item ✓ Implementovány notifikace pro~připomenutí o~návycích.
\end{itemize}

\section{Hlavní funkcionality aplikace}

\subsection{Autentizace}

Aplikace obsahuje systém autentizace s~registrací, přihlášením a~ukládáním přihlašovacích údajů.

\subsection{Správa návyků}

Uživatel může vytvářet, upravovat a~mazat návyky s~vlastními parametry (název, popis, barva, ikona, denní cíl), označovat návyky jako splněné pro~daný den a~zobrazovat progress a~streak.

\subsection{Kalendář}

Kalendář zobrazuje historii plnění návyků pro~jednotlivé dny, umožňuje přidat poznámky k~datům a~zobrazuje vizuální indikace splněných/nesplněných návyků.

\FloatBarrier
\begin{figure}[H]
    \centering
    \begin{subfigure}[t]{0.48\textwidth}
        \centering
        \includegraphics[width=0.85\textwidth]{kalendar.png}
        \caption{Kalendář s~historií plnění návyků}
        \label{fig:calendar}
    \end{subfigure}
    \hfill
    \begin{subfigure}[t]{0.48\textwidth}
        \centering
        \includegraphics[width=0.85\textwidth]{kalendarpoznamka.png}
        \caption{Přidání poznámky do~kalendáře}
        \label{fig:calendarNote}
    \end{subfigure}
    \caption{Kalendář s~historií plnění návyků a~přidání poznámky}
    \label{fig:calendarGroup}
\end{figure}

\subsection{Statistiky a Achievements}

Statistiky obsahují týdenní a~měsíční přehledy, grafy pokroku a~počty splnění návyků. Systém achievementů motivuje uživatele (7, 30, 100 dní v~řadě).

\FloatBarrier
\begin{figure}[H]
    \centering
    \begin{subfigure}[t]{0.48\textwidth}
        \centering
        \includegraphics[width=0.85\textwidth]{statistiky.png}
        \caption{Statistiky a~grafy pokroku}
        \label{fig:stats}
    \end{subfigure}
    \hfill
    \begin{subfigure}[t]{0.48\textwidth}
        \centering
        \includegraphics[width=0.85\textwidth]{oceneni.png}
        \caption{Systém achievementů}
        \label{fig:achievements}
    \end{subfigure}
    \caption{Statistiky a~systém achievementů}
    \label{fig:statsGroup}
\end{figure}

\subsection{Personalizace a Notifikace}

Uživatel může přepínat mezi světlým a~tmavým režimem, měnit barvu motivu a~nastavit profil. Systém notifikací umožňuje naplánovat připomenutí pro~návyky.

\FloatBarrier
\begin{figure}[H]
    \centering
    \begin{subfigure}[t]{0.48\textwidth}
        \centering
        \includegraphics[width=0.85\textwidth]{ucet.png}
        \caption{Profil uživatele a~nastavení}
        \label{fig:profile}
    \end{subfigure}
    \hfill
    \begin{subfigure}[t]{0.48\textwidth}
        \centering
        \includegraphics[width=0.85\textwidth]{casovac.png}
        \caption{Časovač pro~návyky s~časovým limitem}
        \label{fig:timer}
    \end{subfigure}
    \caption{Profil uživatele a~časovač}
    \label{fig:profileGroup}
\end{figure}

\section{Uživatelský manuál}

\subsection{První spuštění a přihlášení}

Po spuštění aplikace se zobrazí splash screen s~logem aplikace. Následně se uživatel dostane na~obrazovku přihlášení nebo registrace. Po úspěšném přihlášení se dostane na~hlavní obrazovku.

\FloatBarrier
\begin{figure}[H]
    \centering
    \begin{subfigure}[t]{0.48\textwidth}
        \centering
        \includegraphics[width=0.85\textwidth]{loadingScreen.png}
        \caption{Loading screen s~logem aplikace}
        \label{fig:loading}
    \end{subfigure}
    \hfill
    \begin{subfigure}[t]{0.48\textwidth}
        \centering
        \includegraphics[width=0.85\textwidth]{homeScreen.png}
        \caption{Hlavní obrazovka s~návyky}
        \label{fig:home}
    \end{subfigure}
    \caption{Loading screen a~hlavní obrazovka}
    \label{fig:loadingGroup}
\end{figure}

\subsection{Vytvoření návyku}

Proces přidání nového návyku probíhá ve~třech krocích:

\begin{enumerate}
    \item Na~hlavní obrazovce klikněte na~tlačítko "+" (vpravo nahoře)
    \item Vyplňte formulář (název, popis, barva, ikona, denní cíl)
    \item Volitelně nastavte připomenutí
    \item Klikněte na~"Vytvořit návyk"
\end{enumerate}

\FloatBarrier
\begin{figure}[H]
    \centering
    \begin{subfigure}[t]{0.31\textwidth}
        \centering
        \includegraphics[width=0.85\textwidth]{pridaniNavyku1.png}
        \caption{Krok 1: základní informace}
        \label{fig:addHabit1}
    \end{subfigure}
    \hfill
    \begin{subfigure}[t]{0.31\textwidth}
        \centering
        \includegraphics[width=0.85\textwidth]{pridaniNavyku2.png}
        \caption{Krok 2: výběr ikony}
        \label{fig:addHabit2}
    \end{subfigure}
    \hfill
    \begin{subfigure}[t]{0.31\textwidth}
        \centering
        \includegraphics[width=0.85\textwidth]{pridaniNavyku3.png}
        \caption{Krok 3: dokončení}
        \label{fig:addHabit3}
    \end{subfigure}
    \caption{Proces vytvoření návyku ve třech krocích}
    \label{fig:addHabitGroup}
\end{figure}

\subsection{Základní funkce}

Na~hlavní obrazovce lze kliknutím na~kartu návyku označit návyk jako splněný. V~dolní navigaci lze přepínat mezi obrazovkami pro~zobrazení statistik a~nastavení profilu.

\section{Technické specifikace}

Aplikace vyžaduje Android 5.0 (API level 21) nebo vyšší a~minimálně 50 MB volného místa. Podporuje offline fungování, dark mode, notifikace, profilové fotky a~statistiky s~grafy.

% Závěr (nečíslovaný)
\chapter*{Závěr}
\addcontentsline{toc}{chapter}{Závěr}

Hlavním cílem této závěrečné práce bylo vytvořit plně funkční mobilní aplikaci pro~sledování návyků a~zdraví pomocí Flutter frameworku. Tento cíl byl úspěšně splněn. Aplikace HabitTrack poskytuje uživatelům moderní a~intuitivní nástroj pro~správu jejich denních návyků s~důrazem na~vizualizaci pokroku a~motivaci k~pravidelnému plnění.

V~rámci práce byla implementována kompletní aplikace obsahující systém autentizace, správu návyků, sledování denního plnění, kalendář s~historií, statistiky, systém achievementů, personalizaci, notifikace a~offline fungování s~lokálním ukládáním dat. Aplikace byla vyvinuta s~použitím čisté architektury a~centralizovaného design systému, což usnadňuje údržbu a~budoucí rozvoj. Díky použití Flutter frameworku je aplikace cross-platform a~může být nasazena na~obě hlavní mobilní platformy.

Pro~budoucí rozvoj aplikace by bylo možné implementovat cloud synchronizaci dat mezi zařízeními, sociální funkce pro~sdílení pokroku, export dat do~různých formátů, podporu více jazyků, widgety pro~rychlý přístup, integraci s~health aplikacemi pro~automatické sledování aktivit a~AI doporučení pro~optimalizaci návyků.

\noindent Zdrojový kód aplikace je dostupný na~GitHubu: \href{https://github.com/lubenikmiroslav/zaverecnaPrace}{github.com/lubenikmiroslav/zaverecnaPrace}

% Seznam použitých informačních zdrojů (nečíslovaný)
% Zde končí číslování stránek práce
\makeatletter
\renewcommand{\@biblabel}[1]{[#1]}
\makeatother

\chapter*{Seznam použitých informačních zdrojů}
\addcontentsline{toc}{chapter}{Seznam použitých informačních zdrojů}
\vspace*{-0.5em}

\begin{thebibliography}{99}
    \bibitem{flutter} GOOGLE. \textit{Flutter - Build apps for any screen} [online]. 2024 [cit. 2024]. Dostupné z: \url{https://flutter.dev}
    
    \bibitem{dart} GOOGLE. \textit{Dart programming language} [online]. 2024 [cit. 2024]. Dostupné z: \url{https://dart.dev}
    
    \bibitem{flutter_docs} GOOGLE. \textit{Flutter Documentation} [online]. 2024 [cit. 2024]. Dostupné z: \url{https://docs.flutter.dev}
    
    \bibitem{sqflite} TEKARTIK. \textit{sqflite - SQLite plugin for Flutter} [online]. 2024 [cit. 2024]. Dostupné z: \url{https://pub.dev/packages/sqflite}
    
    \bibitem{shared_preferences} FLUTTER TEAM. \textit{shared\_preferences - Flutter plugin for storing key-value data} [online]. 2024 [cit. 2024]. Dostupné z: \url{https://pub.dev/packages/shared_preferences}
    
    \bibitem{stackoverflow} STACK EXCHANGE INC. \textit{Stack Overflow - Where Developers Learn, Share, \& Build Careers} [online]. 2024 [cit. 2024]. Dostupné z: \url{https://stackoverflow.com}
    
    \bibitem{github} GITHUB INC. \textit{GitHub - Where the world builds software} [online]. 2024 [cit. 2024]. Dostupné z: \url{https://github.com}
    
    \bibitem{w3schools} W3SCHOOLS. \textit{W3Schools Online Web Tutorials} [online]. 2024 [cit. 2024]. Dostupné z: \url{https://www.w3schools.com}
    
    \bibitem{mdn_web_docs} MOZILLA. \textit{MDN Web Docs} [online]. 2024 [cit. 2024]. Dostupné z: \url{https://developer.mozilla.org}
    
    \bibitem{flutter_cookbook} GOOGLE. \textit{Flutter Cookbook} [online]. 2024 [cit. 2024]. Dostupné z: \url{https://docs.flutter.dev/cookbook}
\end{thebibliography}

% Seznam obrázků
\newpage
\addcontentsline{toc}{chapter}{Seznam obrázků}
\listoffigures

\end{document}
