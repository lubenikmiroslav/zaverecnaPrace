\section{Návrh architektury aplikace}

Aplikace je strukturována podle principů čisté architektury s~oddělením vrstev. Struktura projektu je následující:

\subsection{Struktura projektu}

\begin{itemize}
    \item \texttt{lib/screens/} -- obrazovky aplikace (UI vrstva)
    \item \texttt{lib/services/} -- business logika a~služby (databáze, notifikace)
    \item \texttt{lib/widgets/} -- znovupoužitelné widgety
    \item \texttt{lib/styles/} -- centralizované styly a~design systém
\end{itemize}

\subsection{Design Patterns}

V~aplikaci jsou použity následující návrhové vzory:

\begin{itemize}
    \item \textbf{Singleton} -- pro~DatabaseHelper a~NotificationService (zajišťuje jedinou instanci)
    \item \textbf{State Management} -- pomocí StatefulWidget a~setState pro~lokální stav
    \item \textbf{Repository Pattern} -- pro~práci s~databází (abstrakce datové vrstvy)
    \item \textbf{Factory Pattern} -- pro~vytváření widgetů a~stylů
\end{itemize}

\section{Návrh databáze}

Databáze obsahuje následující tabulky:

\subsection{Schéma databáze}

\begin{itemize}
    \item \texttt{users} -- informace o~uživatelích (id, email, nickname, password\_hash, profile\_photo\_path)
    \item \texttt{habits} -- definice návyků (id, user\_id, name, description, color, icon, daily\_target, atd.)
    \item \texttt{habit\_logs} -- záznamy o~plnění návyků (id, habit\_id, date, completed)
    \item \texttt{calendar\_notes} -- poznámky k~datům (id, user\_id, date, note)
    \item \texttt{achievements} -- odemčené achievementy (id, user\_id, habit\_id, type, unlocked\_at)
\end{itemize}

\subsection{Migrace databáze}

Databáze podporuje migrace pro~plynulé aktualizace schématu. Při změně verze databáze se automaticky provedou potřebné změny (přidání sloupců, vytvoření nových tabulek).

\section{Návrh uživatelského rozhraní}

UI bylo navrženo s~důrazem na:

\begin{itemize}
    \item Minimalistický a~moderní design
    \item Konzistentní barevné schéma (gradient Pink → Orange)
    \item Animace pro~lepší uživatelský zážitek
    \item Dark mode podpora
    \item Responzivní layout pro~různé velikosti obrazovek
\end{itemize}

\section{Implementační postupy}

\subsection{Inicializace aplikace}

Aplikace začíná v~souboru \texttt{main.dart}, kde se inicializují klíčové služby (databáze, notifikace) a~načítají uživatelská nastavení.

\begin{lstlisting}[caption=Inicializace aplikace v~main.dart, label=lst:main]
void main() async {
  WidgetsFlutterBinding.ensureInitialized();
  await DatabaseHelper.instance.database;
  await NotificationService.instance.initialize();

  final prefs = await SharedPreferences.getInstance();
  final savedEmail = prefs.getString('user_email');
  final isDarkMode = prefs.getBool('dark_mode') ?? false;
  final themeColor = prefs.getString('theme_color') ?? '#009688';

  runApp(HabitTrackApp(
    isLoggedIn: savedEmail != null,
    isDarkMode: isDarkMode,
    themeColor: themeColor,
  ));
}
\end{lstlisting}

\subsection{Databázová vrstva}

Databázová vrstva je implementována pomocí Singleton patternu v~třídě \texttt{DatabaseHelper}. Všechny databázové operace jsou asynchronní a~používají SQL dotazy.

\begin{lstlisting}[caption=Implementace Singleton patternu pro~DatabaseHelper, label=lst:database]
class DatabaseHelper {
  static final DatabaseHelper instance = DatabaseHelper._init();
  static Database? _database;

  DatabaseHelper._init();

  Future<Database> get database async {
    if (_database != null) return _database!;
    _database = await _initDB('habittrack.db');
    return _database!;
  }

  Future<Database> _initDB(String filePath) async {
    final dbPath = await getDatabasesPath();
    final path = join(dbPath, filePath);
    return await openDatabase(path, version: 6, 
                              onCreate: _createDB, 
                              onUpgrade: _onUpgrade);
  }
}
\end{lstlisting}

\subsection{Správa návyků}

Hlavní obrazovka zobrazuje seznam návyků uživatele s~možností označit je jako splněné. Při dokončení návyku se zobrazí confetti efekt pro~vizuální zpětnou vazbu.

\begin{lstlisting}[caption=Načítání návyků a~jejich statistik, label=lst:habits]
Future<void> _loadHabits() async {
  final data = await DatabaseHelper.instance.getHabits(userId);
  final todayStr = DateTime.now().toIso8601String().split('T').first;
  
  Map<int, int> todayCount = {};
  Map<int, int> streaks = {};
  
  for (var habit in data) {
    final habitId = habit['id'] as int;
    final count = await DatabaseHelper.instance
        .getDailyCompletionCount(habitId, todayStr);
    todayCount[habitId] = count;
    
    final streak = await DatabaseHelper.instance
        .getHabitStreak(habitId);
    streaks[habitId] = streak;
  }
  
  setState(() {
    habits = data;
    completedTodayCount = todayCount;
    habitStreaks = streaks;
  });
}
\end{lstlisting}

\subsection{Stylování a~design systém}

Aplikace používá centralizovaný design systém s~oddělenými soubory pro~barvy, gradienty, textové styly a~dekorace. To umožňuje snadnou údržbu a~konzistenci vzhledu.

\begin{lstlisting}[caption=Definice barev v~design systému, label=lst:colors]
class AppColors {
  // Primární barvy
  static const Color primaryOrange = Color(0xFFFF9800);
  static const Color primaryPink = Color(0xFFE91E63);
  static const Color primaryPurple = Color(0xFF9C27B0);
  
  // Barvy pro~návyky
  static const Color habitPink = Color(0xFFE91E63);
  static const Color habitRed = Color(0xFFF44336);
  static const Color habitOrange = Color(0xFFFF9800);
  // ... další barvy
}
\end{lstlisting}

\subsection{Animace a~efekty}

Aplikace obsahuje různé animace pro~zlepšení uživatelského zážitku:
\begin{itemize}
    \item AnimatedGradientBackground -- animované gradientní pozadí
    \item ConfettiEffect -- confetti při dokončení návyku
    \item BounceAnimation -- bounce efekt pro~seznamy
    \item ScaleOnTap -- scale efekt při kliknutí
    \item SkeletonLoader -- skeleton loading screens
\end{itemize}

\section{Způsoby testování}

Aplikace byla testována na~několika úrovních:

\subsection{Manuální testování}

Aplikace byla manuálně testována na:
\begin{itemize}
    \item Android emulátoru (různé verze Android)
    \item Fyzickém Android zařízení
\end{itemize}

\subsection{Testované funkcionality}

\begin{itemize}
    \item Autentizace (registrace, přihlášení)
    \item Správa návyků (vytváření, úprava, mazání)
    \item Označení návyku jako splněného
    \item Zobrazení statistik a~grafů
    \item Notifikace a~připomenutí
    \item Dark mode přepínání
    \item Personalizace (barvy, profilová fotka)
\end{itemize}

\section{Řešení problémů}

Během vývoje byly řešeny následující problémy:

\subsection{Přetečení layoutu na~kartách návyků}

\textbf{Problém:} Ikonky a~text se překrývaly na~kartách návyků při menších obrazovkách.

\textbf{Řešení:} Zvýšena výška karet a~upraveno rozložení prvků s~lepším paddingem.

\subsection{Bílá obrazovka před splash screenem}

\textbf{Problém:} Na~Android zařízeních se zobrazovala bílá obrazovka před načtením Flutter splash screenu.

\textbf{Řešení:} Upraveny native Android launch screen soubory (\texttt{launch\_background.xml}) pro~zobrazení gradientního pozadí.

\subsection{Type mismatch při dark mode}

\textbf{Problém:} Některé barvy byly definovány jako funkce vyžadující BuildContext, ale používaly se jako statické hodnoty.

\textbf{Řešení:} Rozděleny barvy na~statické (pro~gradient pozadí) a~context-aware (pro~karty a~text).
