\section{Flutter Framework}

Flutter je open-source UI framework vyvinutý společností Google pro~vytváření nativních aplikací pro~mobilní, webové a~desktopové platformy z~jediného kódu. Flutter používá programovací jazyk Dart.

\subsection{Výhody Flutteru}

Volba Flutter frameworku byla motivována následujícími výhodami:
\begin{itemize}
    \item \textbf{Cross-platform vývoj} -- jedna codebase pro~Android, iOS, Web a~Desktop
    \item \textbf{Vysoký výkon} -- kompilace do~nativního kódu (AOT compilation)
    \item \textbf{Bohatá widget knihovna} -- Material Design a~Cupertino widgety
    \item \textbf{Hot reload} -- rychlý vývoj s~okamžitou aktualizací změn
    \item \textbf{Silná komunita} -- rozsáhlá dokumentace a~podpora
    \item \textbf{Bezplatný a~open-source} -- žádné licenční poplatky
\end{itemize}

\section{Dart programovací jazyk}

Dart je objektově orientovaný programovací jazyk vyvinutý společností Google. Je navržen pro~vývoj aplikací, které běží na~klientovi i~serveru.

\subsection{Klíčové vlastnosti Dartu}

\begin{itemize}
    \item Statická typová kontrola pro~lepší bezpečnost kódu
    \item Asynchronní programování pomocí \texttt{async/await}
    \item Garbage collection pro~automatickou správu paměti
    \item Mixiny pro~vícenásobnou dědičnost
    \item Streamy pro~zpracování datových toků
\end{itemize}

\section{SQLite databáze}

SQLite je lehká relační databázová knihovna implementovaná jako knihovna C. V~aplikaci je použita přes balíček \texttt{sqflite} pro~lokální ukládání dat.

\subsection{Výhody SQLite}

Volba SQLite byla motivována:
\begin{itemize}
    \item \textbf{Offline fungování} -- data jsou uložena lokálně na~zařízení
    \item \textbf{Rychlý přístup} -- přímý přístup k~datům bez síťového připojení
    \item \textbf{Snadná implementace} -- jednoduché SQL dotazy
    \item \textbf{Malá velikost} -- minimální nároky na~paměť
    \item \textbf{Bez serveru} -- nevyžaduje databázový server
\end{itemize}

\section{SharedPreferences}

SharedPreferences je mechanismus pro~ukládání jednoduchých datových typů (String, int, bool) v~key-value formátu. V~aplikaci je použit pro~ukládání uživatelských nastavení (dark mode, theme color, login status).

\section{Flutter Local Notifications}

Balíček \texttt{flutter\_local\_notifications} umožňuje zobrazovat lokální notifikace na~zařízení. V~aplikaci je použit pro~připomenutí uživatelům o~plnění návyků v~nastavený čas.

\section{FL Chart}

FL Chart je knihovna pro~vytváření grafů a~vizualizací v~Flutter aplikacích. V~aplikaci je použita pro~zobrazení statistik a~pokroku uživatelů v~různých formátech (line charts, bar charts).

\section{Image Picker}

Balíček \texttt{image\_picker} umožňuje uživatelům vybrat obrázky z~galerie nebo pořídit fotografie pomocí kamery. V~aplikaci je použit pro~výběr profilové fotky uživatele.

\section{Path Provider}

Balíček \texttt{path\_provider} poskytuje přístup k~systémovým cestám pro~ukládání souborů. V~aplikaci je použit pro~ukládání profilových fotografií uživatelů.

\section{Zdůvodnění výběru technologií}

Výběr technologií byl proveden s~ohledem na~požadavek na~cross-platform kompatibilitu, potřebu offline fungování aplikace, snahu o~moderní a~atraktivní uživatelské rozhraní, dostupnost a~kvalitu dokumentace a~aktivitu komunity a~podporu.

Všechny použité technologie jsou open-source a~bezplatné, což umožňuje volné použití a~distribuci aplikace.
