\section{Flutter Framework}

Flutter je open-source UI framework vyvinutý společností Google pro~vytváření nativních aplikací pro~mobilní, webové a~desktopové platformy z~jediného kódu. Flutter používá programovací jazyk Dart a~kompiluje aplikace do~nativního kódu, což zajišťuje vysoký výkon.

\subsection{Zdůvodnění výběru}

Volba Flutter frameworku byla motivována možností vytvářet aplikace pro~více platforem z~jediného kódu, vysokým výkonem díky kompilaci do~nativního kódu, bohatou widget knihovnou, rychlým vývojem pomocí hot reload a~silnou komunitou s~rozsáhlou dokumentací.

\section{Dart programovací jazyk}

Dart je objektově orientovaný programovací jazyk vyvinutý společností Google. Je navržen pro~vývoj aplikací, které běží na~klientovi i~serveru. Dart poskytuje statickou typovou kontrolu, asynchronní programování pomocí \texttt{async/await}, automatickou správu paměti (garbage collection) a~podporu pro~mixiny a~streamy.

\section{SQLite databáze}

SQLite je lehká relační databázová knihovna implementovaná jako knihovna C. V~aplikaci je použita přes balíček \texttt{sqflite} pro~lokální ukládání dat. SQLite byla zvolena pro~svou jednoduchost, rychlost, offline fungování a~minimální nároky na~paměť.

\section{SharedPreferences}

SharedPreferences je mechanismus pro~ukládání jednoduchých datových typů (String, int, bool) v~key-value formátu. V~aplikaci je použit pro~ukládání uživatelských nastavení jako je dark mode, barva motivu a~stav přihlášení.

\section{Flutter Local Notifications}

Balíček \texttt{flutter\_local\_notifications} umožňuje zobrazovat lokální notifikace na~zařízení. V~aplikaci je použit pro~připomenutí uživatelům o~plnění návyků v~nastavený čas. Tento balíček byl zvolen pro~svou jednoduchost a~dobrou podporu na~Android i~iOS platformách.

\section{FL Chart}

FL Chart je knihovna pro~vytváření grafů a~vizualizací v~Flutter aplikacích. V~aplikaci je použita pro~zobrazení statistik a~pokroku uživatelů v~různých formátech (line charts, bar charts). Knihovna byla zvolena pro~svou flexibilitu a~možnost vytvářet moderní a~atraktivní grafy.

\section{Image Picker}

Balíček \texttt{image\_picker} umožňuje uživatelům vybrat obrázky z~galerie nebo pořídit fotografie pomocí kamery. V~aplikaci je použit pro~výběr profilové fotky uživatele. Balíček poskytuje jednoduché API a~dobrou podporu pro~obě platformy.

\section{Path Provider}

Balíček \texttt{path\_provider} poskytuje přístup k~systémovým cestám pro~ukládání souborů. V~aplikaci je použit pro~ukládání profilových fotografií uživatelů na~správné místo v~systému souborů.

\section{Zdůvodnění výběru technologií}

Výběr technologií byl proveden s~ohledem na~požadavek na~cross-platform kompatibilitu, potřebu offline fungování aplikace, snahu o~moderní a~atraktivní uživatelské rozhraní, dostupnost a~kvalitu dokumentace a~aktivitu komunity a~podporu. Všechny použité technologie jsou open-source a~bezplatné, což umožňuje volné použití a~distribuci aplikace.
